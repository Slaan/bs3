\documentclass[pdftex,10pt,a4paper]{article}
\usepackage[utf8]{inputenc}
\usepackage[german]{babel}
\usepackage{wrapfig}
\usepackage{listings}
\usepackage{amsfonts}
\usepackage{fontenc}
\usepackage{amssymb}
\usepackage{graphicx}
\begin{document}

\title{BS Praktikum 3}
\author{Alex Mantel, Daniel Hofmeister}
\date{\today}
\maketitle
\newpage

\tableofcontents
\newpage

\section{Visualisierung der Strukturen und Parameter}
        
vmem\_struct
  vmem\_adm\_struct
  pt\_struct
    pt\_enty

\section{Vereinfachung der Bitmap}
Anstatt der Bitmap l\"asst sich eine Variable verwenden. Diese wird bei der Allokation der ersten 16 Pages verwendet, um bei jeder einzelnen Allokation eine die Page in eine Frame an der Stelle der Variable zu laden. Anschlie{\ss}end wird die Variable inkrementiert. Wenn die Variable bzw. der Z\"ahler den Wert 15 erreicht, soll f\"ur weitere Allokationen ein Ersetzungsverfahren verwendet werden.

\section{Skizze des Kommunikationsrahmen}


\section{Wie entscheiden wir die geforderte Seite?}

\section{Wie finden die Zugriffe auf die Pagefile mit fwrite und fread statt?}

\section{Aufrufreihenfolge der Funktionen in mmanage.h}

\section{Sukzessive Sequenzdiagramm}

\section{Welche Funktionen setzen die Flags Present, Dirty und Used?}

\bibliographystyle{alpha}
\bibliography{./references}
\end{document}
